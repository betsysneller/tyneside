\documentclass[a4paper,12pt]{letter}
\usepackage{default}
\usepackage[utf8]{inputenc}
\usepackage[T1]{fontenc}


% Some of the article customisations are relevant for this class

%\name{} % To be used for the return address on the envelope
\signature{Joel C. Wallenberg} % Goes after the closing (ie at the end of the letter, with space for a signature)

\address{Joel C. Wallenberg \\ joel.wallenberg@newcastle.ac.uk \\ Lecturer in the History of the English Language \\ P.I., Centre for Behaviour and Evolution \\ +44-(0)191-208-7366}
% Alternatively, these may be set on an individual basis within each letter environment.

%\makelabels % this command prints envelope labels on the final page of the document

\begin{document}
\begin{letter}{\textbf{Application to Join CPT Aboriginal Justice Delegation, August 2014}}

\opening{Dear Christian Peacemaker Teams,} % eg Hello.

The decision to ask you for a place in your upcoming Aboriginal Justice delegation comes after a long period of discernment about what course of action God is calling me to, in order to work against ever increasing violence and injustice in the world. A few years ago when Arthur Gish died, I read an article about his life and began to read his writings and follow CPT's activities. I also read Ron Sider's speech from 1984, which floored me and spoke to me deeply; I've reread it twice since then, and I continue to feel that it is a call to action for Quakers as much as to the Anabaptists he spoke to. During the years since I first became aware of CPT, I've felt a leading towards becoming involved myself, which I have since tested and retested in worship.

I regret that I will not bring many pre-existing skills in nonviolent direct action to the delegation, though I am extremely interested in gaining training in nonviolent direct action and hope that the delegation will be a beginning of that process. I have done considerable reading on the subject and discussed it with sone long-time activists (notably George Lakey and Carolyn McCoy of Earth Quaker Action Team), so I have some understanding of what it entails, and it is clear to me that I need proper training in order to take part in such actions. I do have considerable academic skills, and I hope to write extensively about my experiences on the delegation, and circulate journal entries and other written pieces both within Britain Yearly Meeting of Quakers, various American Quaker organizations, and more widely via the use of social media such as Twitter.

In short, I feel I'm being led by what Quakers call the Inward Christ to two particular actions: first, to gain some training in nonviolent direct action and peacemaking, and secondly, to risk %rather than let others do so for me.




\closing{Yours in the Spirit,} % eg Regards,

%\cc{} % people this letter is cc-ed to
%\encl{} % list of anything enclosed
%\ps{} % any post scriptums. ``PS'' labels must be put in manually

\end{letter}
\end{document}
