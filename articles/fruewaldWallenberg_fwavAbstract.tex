% a percent sign ("%") designates the rest of the line as a comment
% define document, don't change this
\documentclass[a4paper,aps,prl,12pt,tightenlines,superscriptaddress]{revtex4}
\usepackage[top=72pt, bottom=72pt, left=72pt, right=72pt]{geometry}
%\usepackage{pslatex}
%\usepackage{fullpage}
\usepackage{natbib}
\usepackage[utf8]{inputenc}
\usepackage[T1]{fontenc}
\usepackage{gb4e}
\bibpunct{(}{)}{;}{a}{}{,}
%%%%%%%%%% TITLE %%%%%%%%%%
\title{}%Enter title in between these brackets

%%%%%%%%%% AUTHOR(S) AND AFFILIATION(S) %%%%%%%%%%

\begin{document}

\begin{center} \textbf{Optionality is Stable Variation is Competing Grammars}  \end{center}


\noindent This is a strictly Minimalist proposal, in the sense of \citep[][inter alia]{chomsky1993, chomsky2001}.
It takes an issue which has frequently been brute-force encoded in the grammar with little explanatory content, namely optionality, and shows how it can be fully explained as an effect of the interface between the narrow grammar and general properties of the linguistic, cognitive, and social systems in which the derivation is produced.
Our proposal is also not to postulate a set of \textsl{ad hoc} interface constraints; the phenomenon which appears to be optionality in the phonology and syntax is an interface \textsl{effect}, a by-product of how the derivation must relate to independently motivated linguistic and extralinguistic structures.

In this study, we consider four case studies, two syntactic and two morphophonological, which have been described either as diachronically stable variation or as optional operations: \textsl{-in}/\textsl{-ing} variation, Heavy NP Shift (or DP extraposition), [t]/[d]-deletion, and English topicalization (object DP fronting).
Each case involves a probabilistic (i.e. ``optional'') alternation between surface variants, and we show that they are all best understood as instances of an independently attested phenomenon, the alternation between linguistic variants which is observed during a language change in progress.
This is the change-related alternation which is usually called ``competing grammars'' \citep[][]{kroch1989}, which can be defined as the occurrence of alternative atomic units/rules of grammar in a single speaker's inventory, and is usually diachronically unstable.
In fact, the Minimalist research program demands that such a reduction be attempted, and the state of the art in the current  theory of functional heads and movement already predicts that syntactic optionality is the same as competing grammars (i.e. morphosyntactic doublets), though few researchers to date have noticed this \citep[with the exception of][]{kroch1994}.
We also show that even though grammatical optionality is really the same phenomenon as a language change in progress, the competing grammars situation can become diachronically stable under a very specific (and perhaps rare) set of circumstances: when the two variants have partially specialized along a continuous dimension of use with no clear endpoints (e.g. style, phonological heaviness).

\bibliographystyle{linquiry2}
\bibliography{joelrefs}

\end{document}