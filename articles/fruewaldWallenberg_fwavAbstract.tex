% a percent sign ("%") designates the rest of the line as a comment
% define document, don't change this
\documentclass[a4paper,aps,prl,12pt,tightenlines,superscriptaddress]{revtex4}
\usepackage[top=72pt, bottom=72pt, left=72pt, right=72pt]{geometry}
%\usepackage{pslatex}
%\usepackage{fullpage}
\usepackage{natbib}
\usepackage[utf8]{inputenc}
\usepackage[T1]{fontenc}
\usepackage{gb4e}
\bibpunct{(}{)}{;}{a}{}{,}
%%%%%%%%%% TITLE %%%%%%%%%%
\title{}%Enter title in between these brackets

%%%%%%%%%% AUTHOR(S) AND AFFILIATION(S) %%%%%%%%%%

\begin{document}

\begin{center} \textbf{Optionality is Stable Variation is Competing Grammars}  \end{center}


\noindent In this paper, we take an issue which has frequently been brute-force encoded in the grammar with little explanatory content, namely grammatical optionality, and show how it can be fully explained as an effect of the interface between the narrow grammar and general properties of the linguistic, cognitive, and social systems. 
Thus, ours is a strictly Minimalist proposal, in the sense of \citep[][inter alia]{chomsky1993, chomsky2001}; the phenomenon which appears as phonological or syntactic optionality is really the effect of a maximally simple derivation interfacing with other independently motivated linguistic and extralinguistic structures. In showing this, we take three types of variation which have been treated as different phenomena in the literature, optional operations, diachronically stable sociolinguistic variation, and competing grammars in syntactic change \citep[][]{kroch1989}, and show how they can be viewed as a single phenomenon and unified under a general theory of categorical linguistic variation. 
Our proposal is the following: all variation and optionality between categorical variants is a case of competing grammars, or ``doublets'' in the sense of \citet{kroch1994}. 
Stable variation or optionality is simply a subcase of competing grammars in which two variants partially overlap in their contexts of use, but are also partly specialized for use, leading to stochastic behavior in the overlapping context(s). 
This situation of partial specialization is only stable over time under a very specific circumstance: when the variants are specialized along a continuous or ordinal dimension of use which is unbounded in at least one direction (one infinite endpoint).

In this paper, we present four case studies, two syntactic and two morphophonological, which have been described either as diachronically stable variation or as optional operations: \textsl{-in}/\textsl{-ing} variation, Heavy NP Shift (or DP extraposition), [t]/[d]-deletion, and English topicalization.
Each case involves a probabilistic (i.e. ``optional'') alternation between surface variants, and we show that they are all best understood as instances of an independently attested phenomenon, ``competing grammars'', i.e. the alternation between linguistic variants which is observed during a language change in progress \citep[][]{kroch1989}.
If the variants have exactly the same function/meaning in the same linguistic environment, then the variants compete for use and one is replaced by the other over time (language change). 
If the variants specialize for different uses (e.g. meanings, linguistic or extralinguistic contexts), then the choice between variants becomes completely determined and the variants survive over time in their new uses (specialization). 
This much was discussed in \citep{kroch1994}, but Kroch did not address the problem of long-term stable (but probabilistic) variation, as described in \citet{labov1989}, among others. 
We propose that the cases of stable variation are cases of partial specialization, in a situation where complete specialization is not possible because of the nature of the dimension along which the variants have become specialized.

%stuff below needs reorganized
In fact, the Minimalist research program demands that such a reduction be attempted, and the state of the art in the current  theory of functional heads and movement already predicts that syntactic optionality is the same as competing grammars (i.e. morphosyntactic doublets), though few researchers to date have noticed this \citep[with the exception of][]{kroch1994}.
We also show that even though grammatical optionality is really the same phenomenon as a language change in progress, the competing grammars situation can become diachronically stable under a very specific (and perhaps rare) set of circumstances: when the two variants have partially specialized along a continuous dimension of use with no clear endpoints (e.g. style, phonological heaviness).

\bibliographystyle{linquiry2}
\bibliography{joelrefs}

\end{document}