% a percent sign ("%") designates the rest of the line as a comment
% define document, don't change this
\documentclass[a4paper,aps,prl,12pt,tightenlines,superscriptaddress]{revtex4}
\usepackage[top=72pt, bottom=72pt, left=72pt, right=72pt]{geometry}
%\usepackage{pslatex}
%\usepackage{fullpage}
\usepackage{natbib}
\usepackage[utf8]{inputenc}
\usepackage[T1]{fontenc}
\usepackage{gb4e}
\bibpunct{(}{)}{;}{a}{}{,}
%%%%%%%%%% TITLE %%%%%%%%%%
\title{}%Enter title in between these brackets

%%%%%%%%%% AUTHOR(S) AND AFFILIATION(S) %%%%%%%%%%

\begin{document}

\begin{center} \textbf{Optionality is Stable Variation is Competing Grammars}  \\FWAV Workshop \\ Josef Fruehwald (University of Pennsylvania) \\ Joel C. Wallenberg (Newcastle University) \\ joseff@babel.ling.upenn.edu, joel.wallenberg@ncl.ac.uk \end{center}


\noindent In this paper, we take an issue which has frequently been brute-force encoded in the grammar with little explanatory content, namely grammatical optionality, and show how it can be fully explained as an effect of the interface between the narrow grammar and language acquisition on the one hand, and general properties of cognitive and social systems, on the other.
Thus, ours is a strictly Minimalist proposal, in the sense of \citet[][and subsequent]{chomsky1993, chomsky2001}; phenomena which appear as phonological or syntactic optionality are really effects of a maximally simple derivation interfacing with other independently motivated linguistic and extralinguistic structures. In showing this, we analyze three types of variation which have been treated as different phenomena in the literature: optional operations, diachronically stable sociolinguistic variation, and competing grammars. We show how they can be viewed as a single phenomenon and unified under a general theory of categorical linguistic variation. 
Our proposal is the following: all variation and optionality between categorical variants is a case of competing grammars, or ``doublets'' in the sense of \citet{kroch1994}. 
Stable variation or optionality is simply a subcase of competing grammars in which two variants partially overlap in their contexts of use, but are also partly specialized for use, leading to stochastic behavior in the overlapping context(s). 
This situation of partial specialization is only stable over time under a very special circumstance: when the variants are specialized along a continuous dimension of use.% which is unbounded in at least one direction (one infinite endpoint).

In this paper, we present four case studies, two syntactic and two morphophonological, which have been described either as diachronically stable variation or as optional operations: \textsl{-in}/\textsl{-ing} variation, extraposition, [t]/[d]-deletion, and English topicalization.
Each case involves a probabilistic alternation between surface variants, and we show that they are all best understood as instances of an independently attested phenomenon, ``competing grammars'', i.e. the alternation between linguistic variants which is observed during a language change in progress \citep[][]{kroch1989}.
If the variants have exactly the same function/meaning in the same linguistic environment, then the variants compete for use and one is replaced by the other over time (replacement). 
If the variants specialize for different categorical uses (e.g. meanings, (extra-)linguistic  contexts), then the choice between variants becomes completely determined and the variants survive over time in their new uses (specialization). 
This much was discussed in \citet{kroch1994}, but Kroch did not address the problem of long-term stable variation, famously described in \citet{labov1989}, \textsl{inter alia}. 
We propose that cases of stable variation are cases of partial specialization, in a situation where complete specialization is impossible because of a mismatch between categorical variants and an acategorical dimension along which the variants are specializing.
When categorical variants, such as \textsl{-in}/\textsl{-ing}, or C [+topicalize] vs. C [--topicalize], are mapped onto continuous dimensions of specialization, e.g. style, or prosodic heaviness, the outcome can never be complete specialization, because there is no one-to-one mapping between the variants and some set of categories along the dimension of specialization; no such categories exist along this kind of dimension.
 
Furthermore, this result falls out naturally if specialization is understood as the result of a strategy for language acquisition. 
We propose a formal model of acquisition which incorporates the ``Principle of Contrast'', an acquisition strategy proposed in \citet{clark1987, clark1990}, building on the variational acquisition model in \citet{yang2000}.
Once the Principle of Contrast is stated in a mathematically precise way, the different possible outcomes of specialization (complete specialization and stable variation) result automatically from the interaction between the narrow grammar, the formal acquisition model, and the continuous or categorical nature of the dimension of specialization.

\bibliographystyle{linquiry2}
\bibliography{joelrefs}

\end{document}